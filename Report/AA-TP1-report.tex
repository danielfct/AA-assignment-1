\documentclass[a4paper]{article}

%% Language and font encodings
\usepackage[english]{babel}
\usepackage{graphicx}
\usepackage[T1]{fontenc}
\usepackage[utf8]{inputenc}

%% Sets page size and margins
\usepackage[a4paper,top=3cm,bottom=3cm,left=3cm,right=3cm,marginparwidth=1.75cm]{geometry}
%% To use hyperlinks
\usepackage{hyperref}
%% To use colors
\usepackage[usenames,dvipsnames,svgnames,table]{xcolor}

%% To use maths
\usepackage{amssymb}
\usepackage{amsmath}
\usepackage{mathtools}
\usepackage[linesnumbered,ruled,vlined]{algorithm2e} 
\usepackage{titlesec}

%% Command to type a chi character
\DeclareRobustCommand{\rchi}{{\mathpalette\irchi\relax}}
\newcommand{\irchi}[2]{\raisebox{\depth}{$#1\chi$}} 
%% Command to simulate a tab character
\newcommand\tab[1][1cm]{\hspace*{#1}}

%% Set sub of subsections
\setcounter{secnumdepth}{4}
\titleformat{\paragraph}
{\normalfont\normalsize\bfseries}{\theparagraph}{1em}{}
\titlespacing*{\paragraph}
{0pt}{3.25ex plus 1ex minus .2ex}{1.5ex plus .2ex}


\title{\textbf{Aprendizagem Autom\'{a}tica} \\
\large Assignment 1 - Classifiers}

\author{Andrea Mascaretti N52222\and Daniel Pimenta N45404}

\begin{document}
\maketitle
\begin{abstract}
The main goal of this assignment was the parameterization, fitting and comparison of Logistic Regression,
 K-nearest Neighbours and Naive Bayes classifiers.\\ 
The data set used was the banknote authentication which was obtained from the
 \href{https://archive.ics.uci.edu/ml/datasets/banknote+authentication}{UCI machine learning repository}.\\
To achieve our goals we used the Spyder IDE with programming language Python 3.6 and some of its modules 
such as \textit{Panda} to load and manage our data, \textit{NumPy} which is the fundamental package for 
scientific computing with Python and was used for various mathematical calculations and managing multidimensional 
arrays, \textit{sklearn (scikit-learn)} which has efficient tools for data mining and data analysis used in our assignment 
to preprocess data (shuffle, standardize, split), fold our data into k stratified folds, calculate cross validation score and 
fit our data on Logistic Regression and K-nearest Neighbours classifiers. Another imported module was Matplotlib.Pyplot
 which provides a MATLAB-like plotting framework used to build the error plots. 
\end{abstract}

\section{Introduction}
Using the Spyder IDE and programming language Python 3.6 and with a csv file which held data about bank notes authenticity 
we were asked to parameterize, fit and compare three different classifiers: Logistic Regression, K-nearest Neighbours and Naive Bayes. 
For the first two classifiers, we were allowed to import them from a module which was provided by scikit-learn. 
The Naive Bayes classifier was fully implemented for our assignment to fit our data using the best bandwidth we could find and to then later predict the real values
for a test set.\\
The initial data was split into a training set and a test set. The training set was then folded into 5 folds to be used by Cross Validation.
For the Logistic Regression classifier we started with a C value of 1 and incremented to its double for 20 iterations.
For the K value of the K-Nearest Neighbours classifier, we tested k values from 1 to 39 using odd values only. 
In the end, we hope to compare our three classifiers performance using McNemar's test with 95\%  confidence interval 
by comparing their predictions with the real values and pick the best classifier approached in this assignment.\\
Before proceeding with the analysis, we standardised our data. To do so, we centred with respect to the mean and normalised taking into account the standard deviation. It is important to notice that we considered the mean and the standard deviation computed from the training set. Those same transformations were applied afterwards to the test set in order to obtain an estimate of the accuracy of our classifiers. To get an idea of the distribution of our data, we also decided to visualise them by means of a series of scatterplots and density graphs. \includegraphics{"Visual Exploratory Analysis".png}

\section{Classifiers}

\subsection{Logistic Regression}

The logistic regression is a generalised linear model. Before considering
a more classic machine learning framework, we provide a brief introduction
to the main idea behind the model. 

The logistic regression is composed, as every linear model, of three
part:
\begin{itemize}
	\item A sequence of \textit{response variables} $Y_{1},\,\ldots,\,Y_{n}$.
	Such response variables are called random components. The main assumption
	we shall make regarding those variables is that they are all independent
	random variables. Moreover, each one of them will have a distribution
	from the exponential family. Bear in mind that we are not imposing
	that the various $Y_{i}$s are identically distributed.
	\item The \textit{systematic component} is our model. It is a function of
	some predictors (also known as regressors or covariates), linear in
	the parameter and related to the mean of $Y_{i}$.
	\item The \textit{link function} $g\left(\mu_{i}\right)$ will allow us
	to link the two components, allowing us to say that 
	\begin{equation}
	g\left(\mu_{i}\right)=\beta_{0}+\sum_{i=1}^{r}\beta_{i}x_{i}\label{eq:-3}
	\end{equation}
	where $\mu_{i}=\mathbb{E}\left[Y_{i}\right]$. In the machine learning
	framework, it is usually more common to consider the inverse of the
	link function.
\end{itemize}
In our model we will assume the following:
\begin{equation}
Y_{i}\overset{ind}{\sim}\mathsf{Bernoulli}\left(\pi_{i}\right)\label{eq:}
\end{equation}
and therefore we will have that 
\begin{equation}
\mathbb{E}\left[Y_{i}\right]=\pi_{i}.\label{eq:-1}
\end{equation}
Now, in this model we will assume a relationship between the logarithm
of the odds of success for $Y_{i}$ (the log odds or \textit{logit})
and the predictor $\mathbf{x}=\left(1,\,x_{1},\,\ldots,\,x_{r}\right)$.
Mathematically, this entails
\begin{equation}
\ln\left(\frac{\pi}{1-\pi}\right)=\mathbf{\beta^{\prime}\mathbf{x}}\label{eq:-6}
\end{equation}
and $\mathbf{\beta}=\left(\beta_{0},\,\ldots,\,\beta_{r}\right)$.

Rewriting in order to obtain the exponential family form we have
\begin{equation}
\pi^{y}\left(1-\pi\right)^{1-y}=\left(1-\pi\right)\exp\left\{ y\,\ln\left(\frac{\pi}{1-\pi}\right)\right\} ,\label{eq:-2}
\end{equation}
where the term $\ln\left(\frac{\pi}{1-\pi}\right)$ is the natural
parameter of the exponential family and shows why we will implement
the link function
\begin{equation}
g\left(\pi\right)=\ln\left(\frac{\pi}{1-\pi}\right).\label{eq:-4}
\end{equation}

Rewriting \ref{eq:-6}, we obtain
\begin{equation}
\pi\left(x\right)=\frac{\exp\left\{ \mathbf{\beta}^{\prime}\mathbf{x}\right\} }{1+\exp\left\{ \mathbf{\beta}^{\prime}\mathbf{x}\right\} }.\label{eq:-5}
\end{equation}

What we have written is equivalent to saying that 
\begin{equation}
\mathsf{p}\left(\mathcal{C}_{1}|\,\mathbf{x}\right)=\pi\left(\mathbf{x}\right)=\sigma\left(\mathbf{\beta}^{\prime}\mathbf{x}\right)\label{eq:-7}
\end{equation}
where $\sigma\left(\cdot\right)$ is the logistic sigmoid and $\mathcal{C}_{1}$
is to identify class $1$. Notice that it holds $\mathsf{p}\left(\mathcal{C}_{2}|\,\mathbf{x}\right)=1-\mathsf{p}\left(\mathcal{C}_{1}|\,\mathbf{x}\right)=1-\sigma\left(\mathbf{\beta}^{\prime}\mathbf{x}\right)$.

Suppose we are given a dataset $\left\{ \mathbf{x}_{i},\,y_{i}\right\} $
where now $y_{i}\in\left\{ -1,\,1\right\} $ and $\mathbf{x}_{i}\in\mathbb{R}^{r}$
for $i=1,\,\ldots,\,n$. Given our hypotheses, we can write the negative
log-likelihood function as
\begin{equation}
\sum_{i=1}^{n}\ln\left(1+\exp\left\{ -y_{i}\,\left(\mathbf{\beta}^{\prime}\mathbf{x}_{i}\right)\right\} \right).\label{eq:-8}
\end{equation}
To this term we added a $\mathsf{L2}$ penalization term to obtain
a more regularized result, preferring this to its $\mathsf{L1}$ version
as the latter usually provides sparser solutions and we were working
in an environment with only four covariates. This lead to the following
problem:
\begin{equation}
\underset{\mathbf{\beta}}{\min\,}f\left(\mathbf{\beta}\right)=\frac{1}{2}\mathbf{\beta}^{\prime}\mathbf{\beta}+C\,\sum_{i=1}^{n}\ln\left(1+\exp\left\{ -y_{i}\,\left(\mathbf{\beta}^{\prime}\mathbf{x}_{i}\right)\right\} \right),\label{eq:-9}
\end{equation}
where $C\,>\,0$ is a parameter that can be assigned in order to balance
the two terms. To tune it, we relied on a stratified 5-fold cross-validation
on the training set, picking the value that yielded the lowest value.
From a numerical point of view, the trust region Newton method built
in the $\mathsf{ScikitLearn}$ class for $\mathsf{L2}$-logistic regression
was used.

\subsubsection{Optimal C}

Optimal C value found: 256\\
At Cross Validation error value: 0.011\\

To tune the parameter, we relied on a stratified 5-fold cross-validation
on the training set, picking the value that yielded the lowest value.
From a numerical point of view, the trust region Newton method built
in the $\mathsf{ScikitLearn}$ class for $\mathsf{L2}$-logistic regression
was used.
The C parameter acts as a regularising term: the bigger it is, the lesser the regularising impact on the coefficients.
From a bias-variance trade-off point of view, a bigger C term increases the variance and reduces the bias.
 

\includegraphics{"Best C value - L2 Logistic Regression".png}

\subsection{k-Nearest Neighbours}

Suppose that we have some classes $\mathcal{C}_{k}$ such that each
class contains $N_{k}$ points and let $N=\sum_{k}N_{k}$ be the total
number of points. Let us suppose that we are also given a distance
function $\rho:\,\mathbb{R}^{r}\rightarrow\mathbb{R}^{+}$. To classify
a new data point $\mathbf{x}$, we draw a sphere (according to $\rho$)
such that exactly $k$ points, regardless of their class, fall into
it. If we let $V$ be the volume of such sphere and $K_{k}$ be the
number of points of class $k$ contained in it, we see that 
\begin{equation}
\mathsf{p}\left(\mathbf{x}|\,\mathcal{C}_{k}\right)=\frac{K_{k}}{N_{k}V},\label{eq:-10}
\end{equation}
the marginal probability is given by 
\begin{equation}
\mathsf{p}\left(\mathbf{x}\right)=\frac{K}{NV}\label{eq:-11}
\end{equation}
and the prior distributions are 
\begin{equation}
\mathsf{p}\left(\mathcal{C}_{k}\right)=\frac{N_{k}}{N}.\label{eq:-12}
\end{equation}

Therefore, applying Bayes' theorem we have 
\begin{equation}
\mathsf{p}\left(\mathcal{C}_{k}|\mathbf{x}\right)=\frac{\mathsf{p}\left(\mathbf{x}|\,\mathcal{C}_{k}\right)\mathsf{p}\left(\mathcal{C}_{k}\right)}{\mathsf{p}\left(\mathbf{x}\right)}=\frac{K_{k}}{K}.\label{eq:-13}
\end{equation}

\subsubsection{Optimal K}

Optimal K value found: 5\\
At Cross Validation error value: 0.002\\

Comparing the posterior probabilities, we classified each data picking
the maximum \textit{a posteriori}. We consider the usual Euclidean
distance for our distance function and trained only for odds value
of $k$ (between $1$ and $39$), picking the one that yielded the
best result on a Stratified $5$-fold cross validation on our data
set.
We only picked odd numbers to avoid having ties. Alternatively, a rule for breaking ties could have been implemented.

\includegraphics{"Best K value - K-nearest Neighbours".png}

\subsection{Kernel Naive Bayes Classifier}
Let $\mathcal{C}_{k}$ be a class and let $\mathbf{x}$ our data vector.
As done before, we can apply Bayes' theorem to provide the \textit{a
	posteriori} probability of belonging to such class by stating \ref{eq:-13}.
Naive Bayes' classifier is based on the hypothesis that the attributes
of our data are conditionally independent, allowing us to rewrite
the conditional likelihood as 
\begin{equation}
\mathsf{p}\left(\mathbf{x}|\,\mathcal{C}_{k}\right)=\prod_{i=1}^{r}\mathsf{p}\left(x_{i}|\,\mathcal{C}_{k}\right).\label{eq:-14}
\end{equation}
This assumption, that is in fact \textit{naive}, does not usually
hurt too much as we are interested in picking the maximum probability
a\textit{ posteriori}.

Naive Bayes is usually implemented by considering a probability density
function, that is fitted to the feature being taken into account.
In this particular instance, however, the conditional likelihood was
modelled applying kernel density estimation techniques.

If we let $\Omega$ be a nonempty set, a symmetric function (\textit{i.e.}
a function that attains the same value for any permutation of its
arguments) $K:\,\Omega\times\Omega\rightarrow\mathbb{R}$ is called
a positive definite kernel on $\Omega$ if 
\begin{equation}
\sum\sum c_{i}c_{j}K\left(x_{i},\,x_{j}\right)\geq0\label{eq:-15}
\end{equation}
for any $n\in\mathbb{N},\,x_{1},\,\ldots,\,x_{n}\in\Omega,\,c_{1},\,\ldots,\,c_{n}\in\mathbb{R}$.

In this specific setting, we considered the RBF Kernel (also known
as Gaussian kernel), of the form 
\begin{equation}
K_{h}\left(\mathbf{x},\mathbf{x}\right)=\exp\left\{ -\frac{\Vert\mathbf{x}-\mathbf{x^{\prime}}\Vert}{h}\right\} .\label{eq:-16}
\end{equation}

Once the kernel form is fixed, the density for a point $y$ can be
computed by considering 
\[
\rho_{K}\left(y\right)=\sum K\left(\left(y-x_{i}\right)/h\right).
\]

Notice that the bandwidth here acts as a bias-variance trade-off parameter.

To classify the data we considered the logarithm of the posterior
probability, obtaining

\begin{equation}
\ln\left(\mathsf{p}\left(\mathcal{C}_{k}|\mathbf{x}\right)\right)\propto\ln\left(\mathsf{p}\left(\mathcal{C}_{k}\right)\right)+\sum\ln\left(\mathsf{p}\left(\mathbf{x}|\mathcal{C}_{k}\right)\right).\label{eq:-17}
\end{equation}
The prior was fixed simply by taking into accounting the elements
per class and the likelihood by means of the $\mathsf{KernelDensity}$
function in the $\mathsf{ScikitLearn}$ library.

To classify our data into the two categories, we considered 
\begin{equation}
\arg\underset{k\in\left\{ 0,1\right\} }{\max}\ln\left(\mathsf{p}\left(\mathcal{C}_{k}\right)\right)+\sum\ln\left(\mathsf{p}\left(\mathbf{x}|\mathcal{C}_{k}\right)\right).\label{eq:-18}
\end{equation}

\subsubsection{Optimal Bandwidth found}

Optimal Bandwidth found: 0.05\\
At Cross Validation error value: 0.00\\

To set the parameter $h$, once again a $5$-fold stratified cross validation was implemented.
Notice that in this case, allowing for smaller values of $h$ implies reducing the bias and therefore increasing the variance.

\includegraphics{"Best Bandwidth - Naive Bayes".png}

\section{Comparing Classifiers}

After parameterizing and fitting our train data into our classifiers we used our test data to calculate the true error of all
three classifiers.
It was important to split our initial data into training and testing data so we could have an unbiased set to now test our predictions.

\subsection{True Error}

The following results were obtained with a random split seed value of 10182017 and a cross validation random seed value of 52222.

\subsubsection{Logisitic Regression}
	\tab True Negative: 253\\
        	\tab False Positive: 1\\
        	\tab False Negative: 2\\
        	\tab True Positive: 201\\
        	\tab Test Error: 0.0066

\subsubsection{K-nearest Neighbours}
	\tab True Negative: 254\\
        	\tab False Positive: 0\\
        	\tab False Negative: 0\\
        	\tab True Positive: 203\\
        	\tab Test Error: 0.0000

\subsubsection{Naive Bayes}
        \tab True Negative: 254\\
        \tab False Positive: 0\\
        \tab False Negative: 0\\
        \tab True Positive: 203\\
        \tab Test Error: 0.0000

\bigbreak
For comparing the three classifiers performances, we used McNemar's test with a 95\% confidence interval.

\subsection{McNemar's test (with continuity correction)}
Let e01 be the number of examples the first classifier
misclassifies but the second classifies correctly, and e10 be the number of examples the second classifier
classifies incorrectly but the first classifier classifies correctly. The difference divided by the total
follows approximately a chi-squared distribution with one degree of freedom:

\begin{equation}
\frac{(|e_{01} - e_{10}| - 1)^2}{e_{01}+e_{10}}\approx\rchi^2_{1}
\end{equation}

From the predicted classification values of two classifiers we were able to calculate the e01 and e10 values by 
doing a simple loop and comparing both classifications with the real classification value:\\

\begin{algorithm}[H]
 	\KwData{first predictions, second predictions, real values}
 	\KwResult{McNemar's test}
	initialize e01 and e10\;
 	\For{every value in real values}{
		get current value of first predictions\;
		get current value of second predictions\;
  		\If{\(\lvert currentFirstPrediction - value \rvert\) equals 1 and second prediction - value equals 0}{
			increment e01\;
   		}
   		\If{first prediction - value equals 0 and \(\lvert currentSecondPrediction - value \rvert\) equals 1}{
			increment e10\;
   		}
  	}
	\Return $\frac{(|e_{01} - e_{10}| - 1)^2}{e_{01}+e_{10}}$
	\caption{McNemar's test}
\end{algorithm}
\bigbreak

After running the McNemar's test to compare all three classifiers we got the following result:\\
\begin{center}
Logistic Regression VS K-nearest neighbours: 1.333\\
Logistic Regression VS Naive Bayes: 1.333\\
K-nearest neighbours VS Naive Bayes: Not applicable
\end{center}

Which indicates that Logistic Regression and K-nearest neighbours classifiers have similar performance and
also that Logistic Regression and Naive Bayes have a similar performance.
The comparison of K-nearest neighbours and Naive Bayes was not possible with McNemar's test because
they both achieved a 0\% test error. But that could indicate both classifiers have similar performance.

\section{Discussion and Conclusion}
This assignment aimed to help us understand the fundamental principles of three 
classifiers: Logistic Regression, K-nearest Neighbours and Naive Bayes. 
It also allowed us to understand and apply some data management techniques 
such as preprocessing data (Load, Shuffle, Split) and the process of Cross Validation. 
The first step on this assignment was to load and preprocessing the data from a 
file which held data from the UCI machine learning repository about bank notes 
authenticity. The goal of the three classifiers was to predict either a bank note 
was real of fake based on four features (Variance, Skewness, Curtosis, Entropy).
A test of performance of those classifiers was done to compare and eventually 
choose the best and to discuss if any is significantly better than the others.\\
From the McNemar's tests we noticed that Logistic Regression and K-nearest 
Neighbours classifiers performances were similar, the latter being just 
a bit better, but not enough to differentiate their performances. 
The same can be said when comparing Logistic Regression and Naive Bayes classifiers.
Their performances were similar but Naive Bayes was slightly better,
but then again, not enough to differentiate the two.
For the test set we had, the McNemar's test with K-nearest neighbours and Naive Bayes was
not possible because both classifiers had a success rate of 100\% when predicting the classifications. 
  
\clearpage

\section{Bibliography}
\begin{itemize}
\item \href{http://aa.ssdi.di.fct.unl.pt/files/AA-05.pdf}{Lecture 5}, 
	\href{http://aa.ssdi.di.fct.unl.pt/files/AA-05_notes.pdf}{Lecture 5 Notes}, 
	\href{http://aa.ssdi.di.fct.unl.pt/files/AA-06.pdf}{Lecture 6}, 
	\href{http://aa.ssdi.di.fct.unl.pt/files/AA-06_notes.pdf}{Lecture 6 Notes}, 
	\href{http://aa.ssdi.di.fct.unl.pt/files/AA-07.pdf}{Lecture 7}, 
	\href{http://aa.ssdi.di.fct.unl.pt/files/AA-07_notes.pdf}{Lecture 7 Notes}\\
\item \href{http://aa.ssdi.di.fct.unl.pt/files/Tutorials.pdf}{Tutorial Notes}\\
\item \href{http://scikit-learn.org/stable/modules/preprocessing.html#preprocessing}{Scikit-learn Preprocessing docs}\\
\item \href{http://scikit-learn.org/stable/supervised_learning.html#supervised-learning}{Scikit-learn Supervised Learning docs}\\
\item \href{http://scikit-learn.org/stable/model_selection.html#model-selection}{Scikit-learn Model Selection docs}\\
\item Christopher M. Bishop, Pattern Recognition and Machine Learning, Springer, 2006
\item George Casella, Roger L. Berger, Statistical Inference, 2nd Ed., Duxbury Press, 2011
\end{itemize}

\end{document}